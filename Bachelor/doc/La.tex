\documentclass{article}
\usepackage[utf8]{inputenc}

\title{Group 7}
%\author{mahyar.tourchimoghaddam }
\date{February 2022}

\begin{document}

\maketitle

\section{Introduction}
\section{Literature}
In this literature section 10 papers will be review and analyzed for information gathering and eventually possibly tool selection. The structure of this will be that there will be answered 3 question to all paper that are relevant aspects of the research for the student, to be able to perform the project. The paper have been divided in groups based of the 3 issue questions. 
\\The question are as follows:
	\par A: What does the new information give that could be useful for this project?
	\par B: Which technology are highlighted?
	\par C: Which approach are in use in the paper?

\begin{enumerate}

\subsection{RQ1: How to visualize the given data in such a way it helps decision makers?}
	\item  \cite{szafir2021connecting}
        \\A: Gives a good of view on how HRI and dataVis work together with the focus on decision making, which will be useful for this project.
    	\\B: In this given paper the technology is their own framework, so in this case it is not very useful for the student
    	\\C: Their approach is data analysis process, and they mention some design method, which is overview, color and comparing.
	\item \cite{zhang2021interactive}
		\\A: Gives insights in how to improve IoT dataVis which can equal improvement of decision making, with the new information in interactive IoT dataVis Business Intelligence(IIoT-DVBI)
		\\B: The people behind this paper have made simulations based on different aspects and amount of experiments on different kinds of paradigms for data visualization(like decision software kit(DSK), logical thinking study in visual processing(LTVP)), the simulation tool itself is not mentioned.
		\\C: Approaches in use in this paper are the data analytic approach specifically AI and machine learning which are including in the data analytic approach, with this approach they find their results and compare them to some experimental studies to show the effect on data visualization in decision making.
	\item  \cite{qin2020making}
		\\A: Gives info on how to make a good dataVis that are effective and efficient, though the authors survey.
		\\B: The technology that are in use in this particular paper are quite some few, for the illustrations in the paper vega, vega-lite, tableau, DeepEye and more. Other that are mentioned are ggplot2, excel and Microsoft power BI and many more.
		\\C: They list 3 approaches for data visualization(dataVis), these are exact dataVis, approximate dataVis and progressive dataVis. They also introduce an iterative approach, in connection to a pipeline that have automatic capabilities and allows focus on the data visualization life-cycle. 
	\item  \cite{zhu2007measuring} Note! Only had a preview of this paper
		\\A: Gives a definition of how to measure visualization effectively, and points out the the lack of standardizing of visualization measurement.
		\\B: No specific technology found in this paper
		\\C: Literature review is being used in the paper as research
	\item  \cite{protopsaltis2020data}
		\\A: Gives tools and techniques that the student can use, for example the different types of charts and what rules there are on the selection side of things and how to incorporate a pipeline to enable automatic data analysis.
		\\B: The writers of this paper mentions a wide variety of data visualization technology e.g plotly, tableau and thingsboard, especially the aforementioned will be look further into in the specification(down under).
		\\C: Since its a paper that review and analyze different corners of data visualization, such as methods and tools, an working approach on how to data visualize is not present.
	\item \cite{nuamah2017human}
		\\A: Gives an idea to the student on how to incorporate and implement HITM to the project and what to look out for, since the paper mostly focuses on HITM as the most vulnerable point in an IoT system, where it goes though on how to measure it and there after improve and or prevent it.
		\\B: The paper used gartner hype cycle for the graphical presentation, and besides that it used RFID(Radio frequency identification) and WSNs(wireless sensor networks) to enable IoT with the help of the 3 identified paradigms(things-oriented, internet-oriented, semantic-oriented).
		\\C: The approach for this paper is called the neuroergonomics approach, which is the investigation of the human behavior in relation the their performance. This can be use for evaluation, and the student may have the need for it for the user evaluation.
\subsection{RQ2: How to incorporate user-centered design in this project?}
	\item  \cite{akanmu2014user}
		\\A: Gives some information and some insights to how to incorporate UCD to the project.
		\\B: As for the technology of this paper, the one in use is the nine stage design framework, which is for designing visualization, the nine stages are divided into 3 phases (precondition have 3 stages, core have 4 stages and analysis have 2 stages) and allows for multiple iterations.
		\\C: The approach that was used in this paper is the user-centered approach in this instance for infoVis(information visualization) design, where there are an illustration of 6 stages of this approach, these stages are: prototyping, interaction & usability studies, work domain analysis, conceptual development, implementation and debugging.
\subsection{RQ3: What are the advantages and disadvantages of data visualization?}
	\item  \cite{aparicio2015data}  
		\\A: Gives useful insight on dataVis with contribution and trends, where is also comes in on dataVis tools and a bit of history of dataVis.
		\\B: No technology are in use here, since the paper is focuses more on the examination and theory about data visualization.
		\\C: No defined approach for this paper, since the authors are talking about data visualization in general.
	\item  \cite{friendly2008brief} Note! Since the paper is quite long(/short book) and due to time constraints, only selected sections have been focused and study through. The selected sections are abstract, introduction, pre 17th century, 1600-1699, 1950-1975, 1975-present. The others sections have been quickly read through.
		\\A: Gives background information of the development of dataVis, with its milestones through time from pre 17th to 2006. There are also some extra with statistical historiography. The information written in here are good for expanding the knowledge and understanding of dataVis.
		\\B: There are no technology for data visualization that were used in this paper/book, because its just a brief retelling of data visualization history.
		\\C: An approach for developing data visualization is not shown but just mentioned.  
	\item \cite{sadiku2016data}
		\\A: Gives visualization techniques of different graph, plot and charts. Furthermore knowledge about different kind of applications visualization can be used in.
		\\B: The paper does not visualize any data with any given technology, instead some technology are mentioned like tableau, ManyEyes and TOPCAT and others.  
		\\C: Since there are no data visualization happening in this paper neither are any approach utilize, nor any relevant approach highlighted.
		
\end{enumerate}		
	
\subsection{Specification} 
	\subsubsection{Approach}
		Agile:
			\\An agile iterative process with multiple iteration and loops. It will not be with feedback since their is only one developer, if there is feedback or review it will be by the guidance counselor. 
		\\Waterfall:
			\\The waterfall model is never recommended against agile, but with the argumentation of that it is maybe better suited for a 1 student project as soon a the requirements specified, and with iteration since iteration benefits sole developers.
		\\Neuroergonomics approach: 
			\\The neuroergonomics approach could complement the oncoming user evaluation, with its study on human behavior and performance.\\
	\subsubsection{Technology}
		Tableau:
			\\Tableau is a visualization tool that are flexible and fast with a big variety of charts and a wide range of custom visualization. It scripted in R and focuses in business intelligence.
		\\Thingsboard:
			\\Thingsboard is an Iot platform with containing modules and that are open-source. Its capabilities is for example: visualize sensor collected data(from more than one device) and custom Iot dashboards. Programming language for this platform could be java, JavaScript and C++.
		\\Excel:
			\\Excel is a spreadsheet and created by Microsoft. It is simple and easy to utilize, but it is not a build as an visualization tool, though it can be versatile.\\
		\\Libraries:
			\\Matplotlib:
				\\Matplotlib is great with it being a open source library  in python, which make it well documented and therefore easier to learn and get help. Other bonuses about the library is that it is free and is also compatible with c and java. A downside to matplotlib is that it is for low level graph making. 
			\\Plotly:
				\\Plotly is a visualization service that are public online cloud-based. It is accessable in python, R and matlab and more, and it have modules for IoT visualization. 
			\\Ggplot2/plotnine 
				\\ggplot2 is written in R and plotnine is an python implementation of ggplot2. It is an data visualization package and plotting system with the use of components.
				
\subsection{Discussion}
	This section will consist of a discusion of the tool selection, where each tool have been stated with 3 pros and 3 cons.
	\\Since excel is not directly a visualization is will not be further discuss nor neither included in the pros and cons table.
	\\For  the other technology the discussion will be based on the table under. In the case of tableau, it have some good pros, but the cons shine a bit more unfortuneately, because it is business oriented, which means large scale, for that reason it does not fit with a solo project, on top of that there is a not so familiar language so the learning curve is quite steep, with all that is mind, tableau is not the best contenter for this project. For thingsboard it is very well suited to this project by being a Iot platform, the only downside is that it will have to be learn, with that said it is a very good conteder for the project. Matplotlib is good contender with its already found material and how well documented it is, though it is quite old so not that well supported and with it being low level graph making the  oppotinuty for interactive data visualization is not great. Plotly have the posibility for IoT visualization and is good for interactive plots, a negative is that it is online and therefore hard to setup and maybe hard to learn with all its tools, with that in mind it is a great contender. Ggplot2/plotnine is not the most favable technology for the use in the project due to its syntax, but a positive is the use of components which is familiar to the student.
    With all that said thingsboard seems to be the most suitable technology for data visualization in the project.
	
    For some extra the approaches will be discuss as well. By the specification an iterative waterfall approach look like a more surficient choose, but with the relization that review/feedback from the agile approach can be use and combined with the user evaluation where the customer will judge the product to get a desired user design, and in the user evaluation a neuroergonomics approach where the focus will be on the customers behavior based on their performance of the product. The bad thing with the neuroergonomics approach it is based on biological study. So with all that reasoning an iterative agile approach seems to benefit the project more.
    
    \begin{table}[]
        \centering
        \begin{tabular}{c|c|c}
        \hline
        Table & Pros & Cons \\
        \hline
             & \\
             & 
        \end{tabular}
        \caption{Pros and Cons table}
        \label{tab:my_label}
    \end{table}
    
\subsection{Detailed analysis}
	While thingsboard seems to be the most beneficial, plotly have been choosen as the tool to use for the project. It will be argumented why later on. 
	\\Plotly an online visualization service, that can be used in different languages and have the room and oppotunities for specification of IoT visualization. Dash is a framework in python which is created by plotly for the use of making  interactive applications. The name dash is from dashboard, because of its purpose to create dashboards. 
	\\Different from tableau(which is a traditional visualization tool), plotly gives full control to the developer. Against matplotlib, plotly is not that so popular but it is growing and versus ggplot2 both can be use to small and large scale businesses, plotly have a bit higher smartscore but a bit lower user satisfaction score.
	
	Benefits: 
		While made for interactice visualization and with posibilities for IoT centric design, beautiful plots with not many lines of code. It has a simple syntax, so simple in fact that it is possible that a person with no knowlegde in that field of area can make dashboards. With dash being open source the chance for assistens is high. While dash is in python, plotly is compatibel with other languages as well, like R and MATLAB
		
	Challenges: 
        While online can be great with easily accessability, it can also be difficult with aiding helping tools and setup, and the doucamentation is a bit out of date. 
        \\Though there are some few challenges which are cons for plotly, they are affordable and can be overcome, especially with all the positives plotly gives, it is assessed that plotly is the best choice to make for the project, since out of the 3 contenders, matplotlib lacks the too much features in interactive visualization with its low level graphing and thingsboard are too GUI based, in that sense that it is a already made tool for visualization with not much(or any) code related work, which is not something the student is looking after.

\bibliographystyle{ACM-Reference-Format}
\bibliography{bib}

\end{document}